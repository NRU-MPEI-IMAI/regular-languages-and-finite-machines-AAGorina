\documentclass[a4paper]{article}
\usepackage[14pt]{extsizes} 
\usepackage[utf8]{inputenc}
\usepackage[russian]{babel}
\usepackage{setspace,amsmath}
\usepackage{graphicx}
 \usepackage{mathtools}
\usepackage[left=20mm, top=15mm, right=15mm, bottom=15mm, nohead, footskip=10mm]{geometry} 
\begin{document}
\section{Задача 1}
Построить конечный автомат, распознающий язык:
\begin{enumerate}
  \item L = \begin{Bmatrix}
w \in \begin{Bmatrix}
a, b, c
\end{Bmatrix}* | |w|_{c} = 1
\end{Bmatrix}
\\ \includegraphics [scale=0.5]{2022-03-30_17-24-37.png}
  \item L = \begin{Bmatrix}
w \in a, b*| |w|_{a} \leq 2, |w|_{b}\geq 2
\end{Bmatrix}
\\ Рассмотрим как прямое произведение двух автоматов:
\\$|w|_{b}\geq 2$
\\ \includegraphics[scale=0.5]{2022-03-30_17-48-07.png}
\\ $|w|_{a} \leq 2$
\\ \includegraphics[scale=0.5]{2022-03-30_17-54-13.png}
\\ \sum = \begin{Bmatrix}
a, b
\end{Bmatrix}
\\ S = ad
 \\T = <cd, ce, cf>
\\ \includegraphics[scale=0.5]{2022-03-30_21-08-13.png}
\item
  \item  L = \begin{Bmatrix}
w\in \begin{Bmatrix}
a, b
\end{Bmatrix} * | |w|_{a} \ne |w|_{b}
\end{Bmatrix}
\\ Рассмотрим L как $L = Q_{1} \cup Q_{2}$,  где $Q_{1} = \begin{Bmatrix}
w \in \begin{Bmatrix}
a, b
\end{Bmatrix} * | |w|_{a} < |w|_{b}
\end{Bmatrix}$ , 
\\ а $Q_{2} = \begin{Bmatrix}
w \in \begin{Bmatrix}
a, b
\end{Bmatrix} * | |w|_{a} > |w|_{b}
\end{Bmatrix}$
\\ $ Q_{1}$ и $Q_{2}$ не являются регулярными и следовательно L не регулярный и его нельзя описать с помощью конечного автомата.
\item $L = \begin{Bmatrix}
w\in a, b *| ww = www
\end{Bmatrix}$
\\ Если расмотреть относительно длины слова, то |ww| = |www| только в том случае когда $w = \lambda$. L описывает пустые слова. 
\end{enumerate}
\section{Задача 2}
\\ Построить автомат используя прямое произведение.
\begin{enumerate}
  \item $L = \begin{Bmatrix}
w \in \begin{Bmatrix}
a, b
\end{Bmatrix}*| |w|_{a} \ge 2 \wedge |w|_{b} \ge 2
\end{Bmatrix}$
\\ Опишем два языка один их которых задает $|w|_{a} \ge 2$, а второй $|w|_{b} \ge 2$. Их произведение даст нам искомых язык.
\\ Автомат которые описывает язык для которого $|w|_{a} \ge 2$ 
\\ \includegraphics[0.5]{2.1(a).png}
\\ Автомат которые описывает язык для которого $|w|_{b} \ge 2$ 
\\ \includegraphics[0.5]{2.1(b).png}
\\ $\sum \begin{Bmatrix}
a, b
\end{Bmatrix}$
\\ $Q = \begin{Bmatrix}
ad, ae, af, bd, be, bf, cd, ce, cf
\end{Bmatrix}$
\\ $s = <ad>$
\\ $T = <cf>$
\\ Прямое произведение:
\\ \includegraphics[0.5]{2.1(a, b).png}
  \item $L = \begin{Bmatrix}
w \in \begin{Bmatrix}
a, b
\end{Bmatrix}*| |w| \ge 3 \wedge |w| odd
\end{Bmatrix}$
\\ Опишем два языка которые описывают |w| - нечетное и $|w| \ge 3$
\\ \includegraphics[0.45]{2.2(a and b).png}
\\ Прямое произведение:
\\ $\sum \begin{Bmatrix}
a, b
\end{Bmatrix}$
\\ $Q = \begin{Bmatrix}
ae, af, ag, be, bf, bg, ce, cf, cg, de, df, dg
\end{Bmatrix}$
\\ $s = <ae>$
\\ $T = <df>$
\\ Результат прямого произведения. Как видно содержит ветви, не выходящие из начального состояния - можно отбросить.
\\ \includegraphics[0.4]{2.2(a, b)pre.png}
\\ Результат в итоге: 
\\ \includegraphics[0.5]{2.2(a, b)post.png}
  \item $L = \begin{Bmatrix}
w \in \begin{Bmatrix}
a, b
\end{Bmatrix}*| |w|_{a} even \wedge |w|_b divisible 3
\end{Bmatrix}$
\end{enumerate}
\end{document}